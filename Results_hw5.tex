\documentclass[letterpaper,11pt]{article} 
\usepackage{amsmath} 
\usepackage{graphicx}
\usepackage{dcolumn}
\usepackage{bm}
\usepackage[mathlines]{lineno}
\usepackage[spanish]{babel}
\usepackage[utf8]{inputenc}
\usepackage{siunitx}
\graphicspath{ {images/} } 

\usepackage{ragged2e}
\newenvironment{Figure}


\usepackage{multicol,caption}
\usepackage[spanish]{babel}
\usepackage[utf8]{inputenc}
\decimalpoint
\usepackage{authblk}
\usepackage[margin=1in,letterpaper]{geometry} % this shaves off default margins which are too big
\usepackage{cite}

\usepackage[final]{hyperref} % adds hyper links inside the generated pdf file
\usepackage[]{units}
\usepackage{float}
\hypersetup{
	colorlinks=true,       % false: boxed links; true: colored links
	linkcolor=blue,          % color of internal links
	citecolor=blue,        % color of links to bibliography
	filecolor=magenta,      % color of file links
	urlcolor=blue         
}  
\usepackage[margin=1in,letterpaper]{geometry}


\usepackage{fancyhdr}
 
\pagestyle{fancy}
\fancyhf{}
\fancyhead[LE,RO]{Métodos Computacionales}

\begin{document}

\DeclareGraphicsExtensions{.png, .jpg, .pdf, .tiff}

\title{\textbf{Resultados}}

\author[]{\textbf{Juan Nicolás Salazar}}
\affil[]{Departamento de Física, Universidad de los Andes}
\date{3 de Diciembre de 2017}




\maketitle
 \noindent
A continuación se presenta la gráfica del ajuste de los datos por estimación bayesiana de las masas en la fórmula de la velocidad de rotación en una galaxia.

\section{Cuerda}

\begin{figure}[H]
    \centering
    \includegraphics[width=8cm]{curva_ajuste}
    \captionof{figure}{Fit de la curva de rotación de una galaxia}
    \label{f1}
\end{figure}



\end{document}
